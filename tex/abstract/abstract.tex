\documentclass[]{article}

%opening
\title{}
\author{Julian Nowainski}
\date{}
\usepackage[utf8]{inputenc}
\begin{document}

\maketitle

\begin{abstract}
%Kontext
\noindent
In dieser Bachelorarbeit geht es um die menschliche haptische Exploration und Suche von Objekten im 3-dimensionalem Raum unter Ausschluss der visuellen Wahrnehmung, bei welcher untersucht werden soll, wie die haptischen Eigenschaften der Objekte, wie Konturen und Form der Oberfläche, mit den angewandten Suchstrategien zusammenhängen und ob es spezifische Merkmale gibt, die für alle Distraktoren,in diesem Sinne die nicht zu suchenden Objekte, gleich sind.
\\
Der Hintergedanke ist, die haptische Exploration von 3-dimensionalen Objektformen zu erforschen, indem man dabei die Eigenschaften der Oberfläche von Objekten im Bezug zu der angewandten Suchstrategie betrachtet, um effiziente Methoden der haptischen Objekterkennung zu finden. Diese könnten unter anderem bei Robotern für solche Suchaufgaben eingesetzt werden.  
\\\\
Die Untersuchung erfolgt in einem Experiment, in welchem Versuchspersonen ein Objekt, ein Target Stimulus, ertasten und es danach in einem Holzrahmen, dem Modular Haptic Stimulus Board (MHSB), welches mit vielen verschiedenen Stimuli bestückt ist, suchen müssen. Es gibt 5 verschiedene Modelle, die jeweils 3x3cm groß sind und eine Form darstellen. Vorhanden sind Halbkugel, Quadrat, Pyramide, Viertelkreis und ein abge-rundetes Dach. Dabei ist das Target, welches am Anfang ertastet wurde, mehrfach vorhanden. Der Rest des Rahmens ist gefüllt mit Distraktoren und leeren Feldern. Die Daten werden mit einem Handschuh, bestückt mit Vicon Markern, aufgenommen, welcher Informationen über die Handpose sowie die haptischen Eigenschaften wie die Druckstärke und Druckstelle liefert.\\
Für die Auswertung wird ein Klassifikator mit gelabelten Daten aus diesem Experiment trainiert, und untersucht, ob es möglich ist ein Target nur anhand der Daten von der Suche selbst zu klassifizieren. Dies würde Rückschlüsse darauf geben, dass es bestimmte Merkmale bei den Distraktoren gibt, die das Target direkt ausschließen. Weiterhin wird untersucht, ob es möglich ist, den Klassifikator nur anhand der Daten der Distraktoren zu trainieren, und damit die verschiedenen Objekte sicher zu klassifizieren. Dadurch könnte man feststellen, ob der Mensch bei der Suche nur binär zwischen Target und Distraktor unterscheidet, oder ob er genügend verschiedene Merkmale exploriert, welche zusätzliche Informationen zu den verschiedenen Distraktor-Objekten liefern.   
\end{abstract}

\end{document}
