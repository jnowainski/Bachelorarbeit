\documentclass[]{article}

%opening
\title{}
\author{Julian Nowainski}
\date{}
\usepackage[utf8]{inputenc}
\begin{document}

\maketitle

\begin{abstract}
\noindent
Der haptische Sinn ist ein wichtiger Bestandteil der menschlichen Wahr-nehmung und unterstützt uns täglich in den verschiedensten Aufgaben. Dazu gehört auch die explorative Suche, welche meist mit Unterstützung des Sehsinnes verbunden ist. Auch in der Robotik ist diese Art der Wahrnehmung schon angekommen und wird neben den visuellen Aufnahmegeräten unter anderem für die Erkennung und Exploration von Objekten und der Umgebung genutzt. \\
In dieser Bachelorarbeit geht es um die menschliche haptische Exploration unter Ausschluss der visuellen Wahrnehmung, bei welcher untersucht werden soll, wie Suchstrategien mit Objekteigenschaften zusammenhängen und wie sich dabei die Rolle der Objekte in der Suchaufgabe (Target oder Distraktor) auf die Strategie auswirkt.  
\\
\\
Die Untersuchung erfolgt in einem Experiment, in welchem Versuchspersonen mit verbundenen Augen ein Objekt (Stimulus) ertasten und es danach in einem Holzrahmen (Modular Haptic Stimulus Board), welches mit vielen verschiedenen Stimuli bestückt ist, suchen müssen. Dabei ist das Target, welches am Anfang ertastet wurde, mehrfach vorhanden. Der Rest des Rahmens ist gefüllt mit Distraktor-Objekten und leeren Feldern. Die Daten werden mit einem Handschuh, bestückt mit Vicon Markern, aufgenommen, welcher Informationen über die Handpose sowie die haptischen Eigenschaften (Druckstärke und Druckstelle) liefert.\\
Ziel ist es, zu untersuchen, ob die Rolle des Objektes bei der Suche Auswirkungen auf einen mittels dieser Daten trainierten Klassifikator hat, und ob man daraus Suchstrategien schließen kann. Dazu zählt die Frage, ob die Distraktoren untereinander ähnliche Daten aufweisen und man dadurch davon ausgehen kann, dass der Mensch diese auf gleichartige Weise exploriert. Außerdem wird untersucht, ob die Form des Zielobjektes ausschlaggebend für die angewandte Suchstrategie ist und welche Merkmale dafür relevant sind. \\
Hintergedanke dabei ist auch, den Trainingsaufwand für die Klassifikatoren zu mindern, falls Suchstrategien sowie Objektrollen ähnliche Merkmale aufweisen die übertragbar auf alle Objektarten sind.
\end{abstract}

\end{document}
