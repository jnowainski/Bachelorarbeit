\documentclass[]{article}

%opening
\title{INSERT SUPER AWESOME THESiS TITLE HERE}
\author{Julian Nowainski}
\date{}
\usepackage[utf8]{inputenc}
\begin{document}

\maketitle

\begin{abstract}
\noindent
Der haptische Sinn ist ein wichtiger Bestandteil der menschlichen Wahr-nehmung und unterstützt uns täglich in den verschiedensten Aufgaben. Dazu gehört auch die explorative Suche, welche meist mit Unterstützung des Sehsinnes verbunden ist. Auch in der Robotik ist diese Art der Wahrnehmung schon angekommen und wird neben den visuellen Aufnahmegeräten unter anderem für die Erkennung und Exploration von Objekten und der Umgebung genutzt. \\
In dieser Bachelorarbeit geht es darum, die menschliche haptische Exploration unter Ausschluss der visuellen Wahrnehmung zu untersuchen, Daten aufzunehmen und auszuwerten, um Zusammenhänge in den Suchstrategien zu finden und die Rolle von Zielobjekt und Distraktor (Objekte, welche nicht das Zielobjekt sind) bei der Suche zu erforschen. \\
\\
Die Untersuchung erfolgt in einem Experiment, in welchem Versuchspersonen mit verbundenen Augen ein Zielobjekt ertasten und es danach in einem Holzrahmen mit vielen verschiedenen Objekten finden müssen. Dabei ist das Zielobjekt mehrfach vorhanden, der Rest des Rahmens ist gefüllt mit Distraktor-Objekten und flachen Feldern. Die Daten werden mit einem Handschuh aufgenommen, welcher Informationen über die Handpose sowie die haptischen Eigenschaften (Druckstärke und Druckstelle) liefert.\\
Ziel ist es, zu untersuchen, ob die Rolle des Objektes bei der Suche Auswirkungen auf einen mittels dieser Daten trainierten Klassifikator hat, und ob man daraus Suchstrategien schließen kann. Dazu zählt die Frage, ob die Distraktoren untereinander ähnliche Daten aufweisen und man dadurch davon ausgehen kann, dass der Mensch alles was nicht dem Zielobjekt entspricht auf gleichartige Weise exploriert. Außerdem wird untersucht, ob die Form des Zielobjektes ausschlaggebend für die angewandte Suchstrategie ist.  

\end{abstract}

\end{document}
