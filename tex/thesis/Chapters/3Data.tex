% Chapter 3

\chapter{Data Generation and Analysis} % Main chapter title

\label{Data} % For referencing the chapter elsewhere, use \ref{Chapter1} 
%----------------------------------------------------------------------------------------

\begin{text}
Dieses Kapitel beinhaltet den größten Teil meiner Arbeit. Alles zum Nachbearbeiten der Daten kommt in dieses Kapitel. Frage zur Struktur, den Forderungen, das Aufnehmen mit MSS und ROS, posptrocessing von Vicon, Synchronisiserung der Vicon Daten mit ROS und das halb-automatische Generieren von labeln sowie eine erste Analyse der fertigen Daten.
\end{text}

\section{Data Structure and Requirements}

\section{Recording}

\section{Postprocessing Vicon Data}

\section{Synchronizing Data and Generating Labels}
\subsection{Synchronizing Glove Data and Vicon Data}
\subsection{Generating Labels}

\section{Analyzing the Data}