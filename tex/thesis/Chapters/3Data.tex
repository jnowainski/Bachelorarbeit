% Chapter 3

\chapter{Data Generation and Analysis} % Main chapter title

\label{Data} % For referencing the chapter elsewhere, use \ref{Chapter1} 
%----------------------------------------------------------------------------------------
This chapter addresses the methods and efforts to tackle the task of data generation as well as labeling the huge amount of data that was recorded.\\
It was the most time-consuming part of this work, since it involved a lot of postprocessing and data cleansing work that was necessary due to the multi-modality of the recording devices and capturing data with different frequencies with various data formats which also were partly unsynchronized. Furthermore methods are explained that were used to label the generated data mostly automatically, based on position data of the hand and the MHSB, as well as a representation of the distribution of the stimuli objects on the frames.

%----------------------------------------------------------------------------------------
\section{Data Structure and Requirements}
The data in this experiment was recorded with multiple devices, including the Vicon system, the 2 parts of the glove as well as 3 cameras, generating side- and top-views. To be able to train a classifier with supervised learning, there were a number of requirements to the data:
\begin{enumerate}
\item Simultaneous data acquisition
\begin{itemize}
\item Capturing all devices at the same time will facilitate upcoming processing steps
\end{itemize}
\item Postprocessing raw data
\begin{itemize}
\item To be able to work with the data, raw data needs to be processed and all files need to be in the same format
\end{itemize}
\item Synchronizing the time-series
\begin{itemize}
\item Delays in the data acquisition and different device frequencies make this step necessary
\end{itemize}
\item Generating the labels
\begin{itemize}
\item For supervised learning, the whole dataset needs to be labeled
\end{itemize}
\end{enumerate}
%----------------------------------------------------------------------------------------
\section{Recording}
To record the data of all devices preferably at the same time and with giving just one start signal, a tool called Multiple Start Synchronizer (MSS) was used. MSS sends a trigger signal to all registered devices which makes them start and stop capturing data.\\
The Vicon and Basler camera data were captured directly within the Vicon Nexus program. For the glove, data was recorded as rosbag consisting of two topics for each part of the glove. Side-view camera images were captured directly as image files.\\
\\
Despite using MSS, there were still delays among the different devices that had to be synchronized separately. 
%----------------------------------------------------------------------------------------
\section{Postprocessing Vicon Data}
The first step in the pipeline was to postprocess the Vicon data. In this procedure, a three-dimensional hand model with marker positions was fitted to an image of the subjects hand, \textcolor{red}{see figure xyz(fitting and reconstruct)}. This model was then used to reconstruct the hand movement during the experiment to approximate marker positions that occurred during gaps in the recording when no camera captured a marker \textcolor{red}{see right side of figure xyz}.\\
\\
The resulting file contains a time-series of the x-,y- and z-position of each marker. Furthermore a file with the joint-angles was generated.
%----------------------------------------------------------------------------------------
\section{Generating Labels}
This section will describe the methodology that was used to generate labels for the recorded data. The challenge was to write a program, that will do most of the work automatically and handle the huge amount of data generated by this experiment.\\
With 7 subjects participating in up to 5 trials each, and a time series containing between 5000 and 7000 data points for each trial, a manual labeling of the data would be too time-consuming. Also having to cope with unsynchronized data due to delays between the modalities would make this task hard to tackle without proper preprocessing. The solution was a program that used the trajectories of the Vicon data to extract objects that were explored during the search experiment and to label them appropriately.

\subsection{Synchronizing Glove and Vicon Data}
A problem that occurred during the acquisition was the delay between starting the Vicon system and the glove recording. Although sending a trigger signal to both systems at the same time, the glove started capturing data approximately 3 to 5 seconds later. Additionally the beginning of the Vicon data had to be cut by 100 to 1000 frames for postprocessing reasons. Fitting the three-dimensional model was only successful when the markers of the first frames had a nearly perfect plane position. As a consequence, an offset had to be defined pointing to the beginning of the Vicon time series because the data only contains a timestamp describing the beginning of the recording. On the other hand the recorded rosbags from the glove came with a timestamp for each sample.\\
Since the frequency of the tactile glove with 150 Hz is lower than the frequency of Vicon with 200 Hz, the trajectory data should be reduced to the length of the tactile glove.\\
\\
Consider we have two time series $V = \{v_{t} \mid t\in T_{V}\}$ and $G = \{g_{t} \mid t\in T_{G}\}$ describing the set for the Vicon data and glove data. The set of timestamps $T_{G}$ was given for the tactile data and consisting of unix time values. For $T_{V}$ the timestamps had to be calculated for each sample from the initial timestamp, the offset and the frequency. \\
To synchronize, a new time series $V' \subset V$ was defined with \begin{center}
$V' = \{v_{t} \mid \forall g_{t_{g}} \in G \exists v_{t_{v}} \in V : t_{v} \geq t_{g} \wedge t_{v} < t_{g+1}, t_{v} \in T_{V}, t_{g} \in T_{G} \}$
\end{center}
This new time series has now equal length to $G$ and each time value $t_{v_{i}} \in T_{V}$ is greater or equal to $t_{g_{i}} \in T_{G}$ and also smaller then the next one  $t_{g_{i+1}} \in T_{G}$.
\subsection{Representing Glove and Objects}
\subsection{Finding Labels}
%----------------------------------------------------------------------------------------
\section{Analyzing the Data}