% Chapter 2

\chapter{Haptic Search Experiment} % Main chapter title

\label{Haptic Search Experiment} % For referencing the chapter elsewhere, use \ref{Haptic Search Experiment} 

%----------------------------------------------------------------------------------------
%In diesem Kapitel wird der Versuch, die benutzte Hardware , %und das komplette Setting von der Aufnahme beschrieben

%----------------------------------------------------------------------------------------
\section{Haptic Search Experiment}
\textit{Include small introduction to haptic search here}

\subsection{Experimental Setup}
The Modular Haptic Stimuli Board (MHSB) makes up the core part of the experiment. It is a setting with two wooden frames that hold stimuli objects. These objects are 3 x 3 cm big wooden blocks, which have a primitive three-dimensional shape on top of it or are just plane. The whole set consists of 360 blocks with 55 different shapes.\\
The first wooden frame can fit 25 objects and is used for learning a target object whereas the second frame has a capacity of 100 objects and is used for searching target objects. The stimuli are statically installed in the frames and not manipulable to allow a focus on just the search task itself.\\
\\
For this experiment, a subset of stimuli was chosen, consisting of 5 different shapes and plane ones \textcolor{red}{[See fig. XXX -> picture of the shapes]}. The target consists of one object and is placed central in the small frame with the rest of the space consisting of plane stimuli. The big frame contains the rest of this subset, where each shape exist 4 to 5 times, including the target. The objects are distributed mostly equally and keep the same rotation throughout the experiment. Only the distribution and the target are changed with each trial.

\subsection{Execution}  
For this experiment, 7 participants were invited and asked to solve a haptic search task while being blindfolded. The participants were 23 to 28 years old and included both genders. All participants were right-handed and have never seen the stimuli objects, so that during the task they never knew how the set of objects looked like and their perception was purely based on the haptic features.\\
Each participant performed on maximal 5 trials, where after each trial, the target was exchanged and the distribution of the stimuli on the big frame was changed. Before the beginning, there were 2 rehearsals to accustom the subjects to the setting. No participant had the same target twice or more and the task was done with just the right hand, while wearing a glove to record relevant data \textcolor{red}{[see section Hardware]}.\\
\\
For the procedure, each participant was given a description of the task \textcolor{red}{[see appendix]}. The task consisted of two parts. \\
The first task was to explore the target object on the small frame
\section{Hardware}

\subsection{Glove}
\subsection{Vicon}
\subsection{Setting}