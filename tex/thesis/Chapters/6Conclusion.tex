% Chapter 6

\chapter{Conclusion} % Main chapter title

\label{Conclusion} % For referencing the chapter elsewhere, use \ref{Chapter1} 
%----------------------------------------------------------------------------------------
In this work three classification problems were proposed to investigate the influence of target and distractor roles using haptic information obtained from a multimodal sensing glove in a haptic search experiment. The data was preprocessed and a low-dimensional representation of the features was calculated by analyzing the behavior of various feature extraction methods as well as hand-crafted features and choosing the most suitable one. It was found that using only the sensors placed in the fingers of the glove yielded the best results. Furthermore a first investigation of the different modalities showed that the model worked best when the data was limited to only the tactile modality. The developed models were able to classify stimuli into one of five shapes: box, sphere, pyramid, quarter and wave for both cases, using only the data where stimuli were target and the case where they were just distractors. \\
For the second experiment, the idea was to see if it is possible to distinguish the role for one object class. The chosen model, a multilayer perceptron with two hidden layers, trained on tactile data using the finger sensors as feature vector, showed high classification results for the whole class spectrum. This outcome showed, that there are differences in the data for the same object regarding whether it was explored in the target role or the distractor role. \\
The initial questions of this work were to see if it would be possible to observe these so called pop-out effects also in the data and additionally answer the question of their applicability in machine learning tasks. Considering now the results of the first two experiments, a first insight was gained to these questions. It was seen that differences for same objects exist regarding their role as well as it was therewith possible to model these experiments as machine learning tasks.\\
The last experiment aimed at finding deeper links between the roles itself. The idea was that instead of classifying the roles of the same object, the model should just learn one stimuli class in a specific role and it should be tested on a different unseen class. The results would give some insight to the question, if there are features in the data specific for the roles. The assumption was that the efficiency of a haptic search executed by humans is also connected to the ability to divide features in general roles, so that it would not be necessary to explore a distractor object fully but instead just long enough to know it is not the target object. It is thereby not mandatory to know the exact distractor stimuli class in order to identify the role. It was found no indication which could reinforce the assumption in the results of the last experiment. However, the model performed better than random which might be correlated to some object specific features that the model learned. If these features are then specific for a role can yet not be said. Additional testing needs to be done here to analyze the learned features further.\\
\\

%For the second investigation a random forest classifier was selected as winner model and performed a classification on the single objects and their roles. The results showed that the model can distinguish the role an object had in the scenario with high reliability. This proved on the one side that one can capture the differences with the multimodal glove and on the other hand brought up an assumption. This was that humans don't identify a distractor as an object itself, but rather as a class containing features that are not corresponding to the target. This is where it is assumed that efficiency comes from since just a small set of features are explored that are sufficient enough to tell the difference between the target and a distractor class but with no specific object knowledge. A further evidence was the result of the first experiment where it was barely possible to distinguish between the distractor objects.\\
%The last investigation should corroborate this theory by showing that the model can separate all distractor objects as one class from the targets.  As the outcome showed, it was possible to a certain degree to classify these classes. \\
Additional inspections could also be to investigate the exact differences in the features that declare the same object as target or distractor. Also the data could be improved by new experiments where the hand size of the participates will be taken into consideration since high variances were measured for the experiments. Having more homogeneous test conditions will most likely result in more similar data. \\
A different direction to investigate in could be to test whether the model could find the target in an online setting faster than the human performing this experiment. \\
The results of this work combined with the further inspections could serve as valuable guidelines for experiments that use tactile data gather from haptic search experiments.