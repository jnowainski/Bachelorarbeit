% Chapter 1

\chapter{Introduction} % Main chapter title

\label{Introduction} % For referencing the chapter elsewhere, use \ref{Chapter1} 

%----------------------------------------------------------------------------------------

% Define some commands to keep the formatting separated from the content 
\newcommand{\keyword}[1]{\textbf{#1}}
\newcommand{\tabhead}[1]{\textbf{#1}}
\newcommand{\code}[1]{\texttt{#1}}
\newcommand{\file}[1]{\texttt{\bfseries#1}}
\newcommand{\option}[1]{\texttt{\itshape#1}}

%----------------------------------------------------------------------------------------
\section{Motivation}
Humans are very skilled when it comes to the task of exploring objects based merely on the haptic feedback they get from touching them. In a setting with multiple objects, a desired object can be found quite fast among the others. Searching for some objects like keys in your bag, or a phone on your nightstand in the dark are some examples of a three-dimensional haptic search task. In such tasks, one searches for an desired object called the target among other objects that are not of interest called distractors. Often, it is a specific feature of an object that makes it stand out among others. These features can be, for an instance, material properties, size, wight or shape \cite{HapticShape}. This is called the pop-out effect where the target feature is then said to be salient with respect to the distractor properties. \\
This phenomenon has not only been aspect of research in the haptic domain, but rather had its beginning in the vision. An example was to find a red dot among green ones, which could be done effortlessly without the need for a throughout search \cite{treisman_gormican_1988}. Furthermore Ledermann and Klatsky found in 1987 \cite{EPs}, that each search strategy consists of a set of patterns of explorations that are called exploratory procedures (EPs). This means that the efficiency of a haptic search is not only based on the salient target feature, but also on the search strategy that is used which can be correlated to the target object.
\section{Goals} \label{Goals}