% Chapter 1

\chapter{Introduction} % Main chapter title

\label{Introduction} % For referencing the chapter elsewhere, use \ref{Chapter1} 

%----------------------------------------------------------------------------------------

% Define some commands to keep the formatting separated from the content 
\newcommand{\keyword}[1]{\textbf{#1}}
\newcommand{\tabhead}[1]{\textbf{#1}}
\newcommand{\code}[1]{\texttt{#1}}
\newcommand{\file}[1]{\texttt{\bfseries#1}}
\newcommand{\option}[1]{\texttt{\itshape#1}}

%----------------------------------------------------------------------------------------
\section{Motivation}
Humans are very skilled when it comes to the task of exploring objects based merely on the haptic feedback they get from touching them. In a setting with multiple objects, a desired object can be found quite fast among the others. Searching for some objects like keys in your bag, or a phone on your nightstand in the dark are some examples of a three-dimensional haptic search task. In such tasks, one searches for an desired object called the target among other objects that are not of interest called distractors. Often, it is a specific feature of an object that makes it stand out among others. These features can be, for an instance, material properties, size, wight or shape \cite{HapticShape}. This is called the pop-out effect where the target feature is then said to be salient with respect to the distractor properties. \\
This phenomenon has not only been aspect of research in the haptic domain, but rather had its beginning in the vision. An example was to find a red dot among green ones, which could be done effortlessly without the need for a throughout search. However, finding a line with a specific rotation in an image with multiple lines of various rotations takes some effort \cite{treisman_gormican_1988}. Furthermore Ledermann and Klatsky found in 1987 \cite{EPs}, that each search strategy consists of a set of patterns of explorations that are called exploratory procedures (EPs). This means that the efficiency of a haptic search is not only based on the salient target feature, but also on the search strategy that is used. This shows the complexity of haptic search tasks and makes it even more interesting that humans can do this with high accuracy and seemingly little effort.  \\
A lot of research was done to investigate human behavior in such tasks, but to this day it was merely investigated what the approaches are to implement such behavior in technical systems such as robots. This work wants to consider two questions with the data recorded in a haptic search experiment with a multimodal glove worn by participants: Is it possible to inspect these pop-out effects based on tactile data and pose-information as well as the question of the applicability of these data in machine learning tasks. 

\section{Goals} \label{Goals}
A first goal of this work was to provide a setting where haptic search tasks could be performed by participants and be recorded with suitable hardware. In section \ref{Haptic Search Experiment} the haptic search experiment will be proposed and discussed starting by the experimental setup, the execution, hardware and the overall setting.\\
After having recorded the data a second goal aimed at generating a data set that was suitable for supervised learning tasks. These included postprocessing tha raw data and generating labels for the individual trials. Section \ref{Data} describes the efforts that were made to reach this goal.\\
The last goal was to use the information collected with the data to perform machine learning tasks to investigate the differences between target- and distractor objects and measure the performance of the developed models. Section \ref{Evaluation} will describe the different experiments and models that were used to address the question for applicability of recorded data from haptic search experiments performed by humans.  
