%%%%%%%%%%%%%%%%%%%%%%%%%%%%%%%%%%%%%%%%%
% Short Sectioned Assignment
% LaTeX Template
% Version 1.0 (5/5/12)
%
% This template has been downloaded from:
% http://www.LaTeXTemplates.com
%
% Original author:
% Frits Wenneker (http://www.howtotex.com)
%
% License:
% CC BY-NC-SA 3.0 (http://creativecommons.org/licenses/by-nc-sa/3.0/)
%
%%%%%%%%%%%%%%%%%%%%%%%%%%%%%%%%%%%%%%%%%

%----------------------------------------------------------------------------------------
%	PACKAGES AND OTHER DOCUMENT CONFIGURATIONS
%----------------------------------------------------------------------------------------

\documentclass[paper=a4, fontsize=11pt]{scrartcl} % A4 paper and 11pt font size

\usepackage[T1]{fontenc} % Use 8-bit encoding that has 256 glyphs
\usepackage{fourier} % Use the Adobe Utopia font for the document - comment this line to return to the LaTeX default
\usepackage[utf8]{inputenc}
\usepackage[english]{babel} % English language/hyphenation
\usepackage{amsmath,amsfonts,amsthm} % Math packages


\usepackage{sectsty} % Allows customizing section commands
\allsectionsfont{\centering \normalfont\scshape} % Make all sections centered, the default font and small caps

\usepackage{fancyhdr} % Custom headers and footers
\pagestyle{fancyplain} % Makes all pages in the document conform to the custom headers and footers
\fancyhead{} % No page header - if you want one, create it in the same way as the footers below
\fancyfoot[L]{} % Empty left footer
\fancyfoot[C]{} % Empty center footer
\fancyfoot[R]{\thepage} % Page numbering for right footer
\renewcommand{\headrulewidth}{0pt} % Remove header underlines
\renewcommand{\footrulewidth}{0pt} % Remove footer underlines
\setlength{\headheight}{0.6pt} % Customize the height of the header

\numberwithin{equation}{section} % Number equations within sections (i.e. 1.1, 1.2, 2.1, 2.2 instead of 1, 2, 3, 4)
\numberwithin{figure}{section} % Number figures within sections (i.e. 1.1, 1.2, 2.1, 2.2 instead of 1, 2, 3, 4)
\numberwithin{table}{section} % Number tables within sections (i.e. 1.1, 1.2, 2.1, 2.2 instead of 1, 2, 3, 4)

\setlength\parindent{0pt} % Removes all indentation from paragraphs - comment this line for an assignment with lots of text

%----------------------------------------------------------------------------------------
%	TITLE SECTION
%----------------------------------------------------------------------------------------

\newcommand{\horrule}[1]{\rule{\linewidth}{#1}} % Create horizontal rule command with 1 argument of height

\title{	
	\normalfont \normalsize 
	\textsc{Julian Nowainski, Universität Bielefeld} \\ [5pt] % Your university, school and/or department name(s)
	%\author{Julian Nowainski} % Your name
	\horrule{0.5pt} \\[0.4cm] % Thin top horizontal rule
	\huge Haptisches Explorationsexperiment \\ % The assignment title
	\large Checkliste
	\horrule{2pt} \\[0.5cm] % Thick bottom horizontal rule
	\date{\vspace{-10ex}} %to remove the date and spaces
}
\begin{document}
	
	\maketitle % Print the title
	
	%----------------------------------------------------------------------------------------
	%	VICON
	%----------------------------------------------------------------------------------------
	
	\section{Vicon}
	\begin{itemize}
		\item Vicon angeschaltet, vorne und hinten
		\item PC angeschaltet
		\item Vicon gestartet, Tuch ausgelegt
		\item Maskiert, kallibriert, Ursprung und Basler ausgewählt bzw. eingestellt
		\item Labor verdunkelt und Lichtverhältnisse stimmen
		\item Ordnerverzeichnis passt
	\end{itemize}
	
	
	%----------------------------------------------------------------------------------------
	%	HANDSCHUH
	%----------------------------------------------------------------------------------------
	
	\section{Handschuh und Cam}
	
	\begin{itemize}
		\item beide Modalitäten angeschlossen
		\item Cyberglove Box angeschaltet
		\item Marker angebracht
		\item Roscore, Listener, Rosrecorder, Treiber gestartet
		\item Topics publishen 
	\end{itemize}
	
	
	%----------------------------------------------------------------------------------------
	%	MSS
	%----------------------------------------------------------------------------------------
	
	\section{Mss}
	\begin{itemize}
		\item MSS gestartet
		\item Modularitäten ausgewählt
		\item Port, Host stimmen
		\item Speicherordner festlegen -> tmp oder mfast
		\item Ping funktioniert
	\end{itemize}
\end{document}
