%%%%%%%%%%%%%%%%%%%%%%%%%%%%%%%%%%%%%%%%%
% Short Sectioned Assignment
% LaTeX Template
% Version 1.0 (5/5/12)
%
% This template has been downloaded from:
% http://www.LaTeXTemplates.com
%
% Original author:
% Frits Wenneker (http://www.howtotex.com)
%
% License:
% CC BY-NC-SA 3.0 (http://creativecommons.org/licenses/by-nc-sa/3.0/)
%
%%%%%%%%%%%%%%%%%%%%%%%%%%%%%%%%%%%%%%%%%

%----------------------------------------------------------------------------------------
%	PACKAGES AND OTHER DOCUMENT CONFIGURATIONS
%----------------------------------------------------------------------------------------

\documentclass[paper=a4, fontsize=11pt]{scrartcl} % A4 paper and 11pt font size

\usepackage[T1]{fontenc} % Use 8-bit encoding that has 256 glyphs
\usepackage{fourier} % Use the Adobe Utopia font for the document - comment this line to return to the LaTeX default
\usepackage[utf8]{inputenc}
\usepackage[english]{babel} % English language/hyphenation
\usepackage{amsmath,amsfonts,amsthm} % Math packages


\usepackage{sectsty} % Allows customizing section commands
\allsectionsfont{\centering \normalfont\scshape} % Make all sections centered, the default font and small caps

\usepackage{fancyhdr} % Custom headers and footers
\pagestyle{fancyplain} % Makes all pages in the document conform to the custom headers and footers
\fancyhead{} % No page header - if you want one, create it in the same way as the footers below
\fancyfoot[L]{} % Empty left footer
\fancyfoot[C]{} % Empty center footer
\fancyfoot[R]{\thepage} % Page numbering for right footer
\renewcommand{\headrulewidth}{0pt} % Remove header underlines
\renewcommand{\footrulewidth}{0pt} % Remove footer underlines
\setlength{\headheight}{0.6pt} % Customize the height of the header

\numberwithin{equation}{section} % Number equations within sections (i.e. 1.1, 1.2, 2.1, 2.2 instead of 1, 2, 3, 4)
\numberwithin{figure}{section} % Number figures within sections (i.e. 1.1, 1.2, 2.1, 2.2 instead of 1, 2, 3, 4)
\numberwithin{table}{section} % Number tables within sections (i.e. 1.1, 1.2, 2.1, 2.2 instead of 1, 2, 3, 4)

\setlength\parindent{0pt} % Removes all indentation from paragraphs - comment this line for an assignment with lots of text

%----------------------------------------------------------------------------------------
%	TITLE SECTION
%----------------------------------------------------------------------------------------

\newcommand{\horrule}[1]{\rule{\linewidth}{#1}} % Create horizontal rule command with 1 argument of height

\title{	
\normalfont \normalsize 
\textsc{Julian Nowainski, Universität Bielefeld} \\ [5pt] % Your university, school and/or department name(s)
%\author{Julian Nowainski} % Your name
\horrule{0.5pt} \\[0.4cm] % Thin top horizontal rule
\huge Haptisches Explorationsexperiment \\ % The assignment title
\large Versuchsanweisung
\horrule{2pt} \\[0.5cm] % Thick bottom horizontal rule
\date{\vspace{-10ex}} %to remove the date and spaces
}
\begin{document}

\maketitle % Print the title

%----------------------------------------------------------------------------------------
%	VERSUCHSBESCHREIBUNG
%----------------------------------------------------------------------------------------

\section{Versuchsbeschreibung}
Bei diesem Versuch müssen Sie mit verbundenen Augen und einem Handschuh eine haptische Suchaufgabe lösen. Es werden Daten sowohl vom Handschuh als auch vom Vicon-System aufgenommen. Das Experiment besteht aus 2 Teilen. \\\\
Der \textbf{erste Teil} der Suchaufgabe beinhaltet das Lernen und Finden eines Objektes in einem Holzbrett, welches aus verschiedenen Formen besteht. Das Zielobjekt ist mehrfach darin enthalten. Sie müssen nicht alle finden, aber so viele sie können. Es wird ein Zeitlimit von \textbf{25 Sekunden} geben. \\
Sie sollen nicht sagen, wo das Objekt ist, oder ob sie es soeben ertastet haben. Merken Sie sich lediglich die ungefähre Position.\\\\
Im \textbf{zweiten Teil} müssen Sie die gefundenen Objekte aus dem Gedächtnis rufen (immer noch mit verbundenen Augen) und auf die Zielobjekte zeigen. Abtasten ist erlaubt, jedoch beachten Sie das geringe Zeitlimit von nurnoch \textbf{5 Sekunden}


%----------------------------------------------------------------------------------------
%	VERSUCHSDURCHFÜHRUNG
%----------------------------------------------------------------------------------------

\section{Versuchsdurchführung}
Insgesamt wird es 2 Durchläufe geben. Die Dauer beträgt ca. eine Stunde

\begin{enumerate}
\item Desinfizieren Sie sich die Hände und ziehen Sie den Handschuh an (Hilfe wird geboten)
\item Als nächstes wird ihnen eine Augenbinde aufgesetzt
\item Legen sie Ihre Hand auf die Startposition und warten Sie auf den Startbefehl
\item Ertasten und merken Sie sich das Objekt auf dem kleinen Brett und suchen Sie es auf dem großen. Sie haben \textbf{25 Sekunden}
\item Warten sie auf den Startbefehl für den zweiten Teil
\item Zeigen Sie auf die Zielobjekte. Sie haben \textbf{5 Sekunden}
\end{enumerate}



%----------------------------------------------------------------------------------------

\end{document}
